\documentclass{article}
\author{Trey Plante}


\title{Notes}

\usepackage[normalem]{ulem}
\usepackage{ebproof}
\usepackage{amsmath,amsthm,amssymb,stmaryrd}
\usepackage[alphabetic]{amsrefs}
\usepackage{hyperref, cleveref}
\usepackage[inline]{enumitem}
\usepackage{mathtools}

\input{macros.tex}

\begin{document}
\maketitle
\tableofcontents

\section{The type theory}

The type theory is a two-level type theory, meaning it contains two notions of equality.

\section{Algorithmic Definitions}

We present an algorithmic procedure for deciding the judgments of the type theory.

\begin{definition}
  Head reduction is defined as follows:

  \[
    \begin{array}{lr}
      \Gamma \triangleright (\lambda x. e)f \rightarrow e[f/x] \\
      \Gamma \triangleright ef \to e'f \qquad \text{ if } \Gamma \triangleright e \to e' \\
      \Gamma \triangleright [\alpha \to e, \beta \to f] \rightarrow e \qquad \text{ if } \Gamma \vdash \alpha\\
      \Gamma \triangleright [\alpha \to e, \beta \to e] \rightarrow f \qquad \text{ if } \Gamma \vdash \beta\\
      \Gamma \triangleright \outB(a) \rightarrow t \qquad \text{ if } \Gamma \triangleright a \uparrow A[\psi \to t] \text{ and } \Gamma \vdash \psi \\
      \Gamma \triangleright \outB(in(a)) \rightarrow a \\
      \Gamma \triangleright \outB(a) \rightarrow a' \qquad \text{ if } \Gamma \triangleright a \rightarrow a' \\
      \Gamma \triangleright \outP(in(a)) \rightarrow a \\
      \Gamma \triangleright \outP(a) \rightarrow a' \qquad \text{ if  } \Gamma \triangleright a \rightarrow a'\\
    \end{array}
  \]
\end{definition}


\begin{definition}
  Term equivalence is defined as follows:

  \[
    \begin{array}{lr}
      \Gamma \triangleright M \Leftrightarrow N : A[\psi \to t] \qquad \text{ if } \Gamma \triangleright \outB(M) \Leftrightarrow \outB(N) : A\\
      \Gamma \triangleright M \Leftrightarrow N : [\alpha] \to A \qquad \text{ if } \Gamma, \alpha \triangleright \outP(M) \Leftrightarrow \outP(N) : A\\
      \Gamma \triangleright M \Leftrightarrow N : (x : \mathbb{I}) \to A \qquad \text{ if } \Gamma, x : \mathbb{I} \triangleright M x \Leftrightarrow N x : A(x)\\
      \Gamma \triangleright M \Leftrightarrow N : (x : \cof) \to A \qquad \text{ if } \Gamma, x : \cof \triangleright M x \Leftrightarrow N x : A(x)\\
    \end{array}
  \]
\end{definition}

\begin{definition}
  Neutral equivalence is defined as follows:

  \[
    \begin{array}{lr}
      \Gamma \triangleright [\alpha \to p_{1}, \beta \to p_{2}] \leftrightarrow q \uparrow A \qquad \text{ if } \Gamma, \alpha \triangleright p_{1} \Leftrightarrow q : A \text{ and } \Gamma, \beta \triangleright p_{2} \Leftrightarrow q : A \\

      \Gamma \triangleright p \leftrightarrow [\alpha \to q_{1}, \beta \to q_{2}] \uparrow A \qquad \text{ if } \Gamma, \alpha \triangleright p \Leftrightarrow q_{1} : A \text{ and } \Gamma, \beta \triangleright p \Leftrightarrow q_{2} : A \\

      \Gamma \triangleright \outB(p) \Leftrightarrow \outB(q) \uparrow A \qquad \text{ if } \Gamma \triangleright p \leftrightarrow q \uparrow A[\psi \to t]\\

      \Gamma \triangleright \outP(p) \Leftrightarrow \outB(q) \uparrow A \qquad \text{ if } \Gamma \triangleright p \leftrightarrow q \uparrow [\alpha] \to A\\

    \end{array}
  \]
\end{definition}

\end{document}
