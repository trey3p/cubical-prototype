\documentclass{article}
\author{Trey Plante}


\title{Notes}

\usepackage[normalem]{ulem}
\usepackage{ebproof}
\usepackage{amsmath,amsthm,amssymb,stmaryrd}
\usepackage[alphabetic]{amsrefs}
\usepackage{hyperref, cleveref}
\usepackage[inline]{enumitem}
\usepackage{mathtools}

\input{macros.tex}

\begin{document}
\maketitle
\tableofcontents

\section{The type theory}

The type theory is a two-level type theory, meaning it contains two notions of equality.

\subsection{UIP Layer}

\subsection{Cubical Layer}

The cubical type theory is parameterized by a rich cofibration logic with the interval structure being that of a distributive lattice.

cofibration specification ...

\begin{definition}
  Partial types are defined by the following rules:


    \[\begin{prooftree}
        \hypo{\Gamma \vdash \A : \sU_{l}}
        \infer1[form]{\Gamma \vdash [\alpha] \to A}
    \end{prooftree}
    \quad
    \begin{prooftree}
        \hypo{\Gamma ,\alpha \vdash a : A}
        \infer1[intro]{\Gamma \vdash \inb(f) : [\alpha] \to \A}
      \end{prooftree}
    \]
    \[
      \begin{prooftree}
        \hypo{\Gamma \vdash f : [\alpha] \to A}
        \infer1[elim]{\Gamma \vdash \outb(f) : A}
    \end{prooftree}
    \quad
    \begin{prooftree}
        \hypo{\Gamma \vdash f : [\alpha] \to \A}
        \infer1[comp]{\Gamma \vdash \outb(f) \equiv a}
    \end{prooftree}
  \]
\end{definition}


\begin{definition}
  Extension types are defined by the following rules:

  \[
     \begin{prooftree}
            \hypo{\Gamma \vdash A : \sU_l} \hypo{\Gamma \vdash \psi \cof} \hypo{\Gamma \vdash t : [\psi] \to \A}
            \infer3[form]{\Gamma \vdash \A[\psi \to t]}
        \end{prooftree}
  \]
  \[
     \begin{prooftree}
            \hypo{\Gamma \vdash a : \A}
            \hypo{\Gamma \vdash \psi \cof}
            \hypo{\Gamma, \psi \vdash t : \A}
            \hypo{\Gamma, \psi \vdash t \equiv a}
            \infer4[intro]{\Gamma \vdash \inb(a) : \A[\psi \to t]}
        \end{prooftree}
      \]
  \[
        \begin{prooftree}
            \hypo{\Gamma \vdash a : \A[\psi \to t]}
            \infer1[elim]{\Gamma \vdash \outb(a) : \A}
        \end{prooftree}
        \quad
        \begin{prooftree}
             \hypo{\Gamma \vdash a : A[\psi \to t]}
             \hypo{\Gamma \vdash \psi}
            \infer2[comp1]{\Gamma \vdash \outb(a) \equiv t}
        \end{prooftree}
        \quad
        \begin{prooftree}
            \hypo{\Gamma \vdash a : A}
            \infer1[comp2]{\Gamma \vdash \outb(\inb(a)) \equiv a}
        \end{prooftree}
  \]

\end{definition}

\begin{definition}
  The elimination form of disjunctions is introduced as follows:


      \begin{prooftree}
            \hypo{\Gamma \vdash A : \sU_{l}} \hypo{\Gamma \vdash_{\bPsi} \alpha \lor \beta} \hypo{\Gamma, \alpha \vdash e_{1} : A} \hypo{\Gamma, \beta \vdash e_{2} : A}
            \hypo{\Gamma, \alpha, \beta \vdash e \equiv f : A}
            \infer5[intro]{\Gamma \vdash [e_{1}, e_{2}]_{\alpha, \beta} : A}
    \end{prooftree}

\end{definition}


\section{Equational Theory}


\begin{definition}
  The equational theory of disjunctions is as follows:

  \[
    \begin{prooftree}
            \hypo{\Gamma \vdash \alpha}
            \infer1[comp1]{\Gamma \vdash [e_{1}, e_{2}]_{\alpha, \beta} \equiv e_{1}}
    \end{prooftree}
    \quad
    \begin{prooftree}
        \hypo{\Gamma \vdash \beta}
        \infer1[comp2]{\Gamma \vdash [e_{1}, e_{2}]_{\alpha, \beta} \equiv e_{2}}
    \end{prooftree}
  \]
  \[
      \begin{prooftree}
        \hypo{\Gamma_{\bPsi} \vdash \alpha \lor \beta}
        \hypo{\Gamma, \alpha \vdash e \equiv f : A}
        \hypo{\Gamma, \beta \vdash e \equiv f : A}
        \infer3[ext]{\Gamma \vdash e \equiv f : A}
    \end{prooftree}
    \quad
    \begin{prooftree}
        \hypo{\Gamma_{\bPsi} \vdash \alpha \lor \beta}
        \hypo{\Gamma \vdash e : A}
        \infer2[eta]{e \equiv [e, e]_{\alpha, \beta} : A}
    \end{prooftree}
  \]
  \[
    \begin{prooftree}
        \hypo{\Gamma \vdash \alpha \Leftrightarrow \gamma}
        \hypo{\Gamma \vdash \beta \Leftrightarrow \delta}
        \hypo{\Gamma, \alpha \vdash e_{1} \equiv f_{1} : A}
        \hypo{\Gamma, \beta \vdash e_{2} \equiv f_{2} : A}
        \infer4[cong]{\Gamma \vdash [e_{1}, e_2]_{\alpha, \beta} \equiv [f_{1}, f_{2}]_{\gamma, \delta} : A}
    \end{prooftree}
  \]



\end{definition}





\section{Algorithmic Definitions}

We present an algorithmic procedure for deciding the judgments of the type theory.

\begin{definition}
  Head reduction is defined as follows:

  \[
    \begin{array}{lr}
      \Gamma \triangleright (\lambda x. e)f \rightarrow e[f/x] \\
      \Gamma \triangleright ef \to e'f \qquad \text{ if } \Gamma \triangleright e \to e' \\
      \Gamma \triangleright [\alpha \to e, \beta \to f] \rightarrow e \qquad \text{ if } \Gamma \vdash \alpha\\
      \Gamma \triangleright [\alpha \to e, \beta \to e] \rightarrow f \qquad \text{ if } \Gamma \vdash \beta\\
      \Gamma \triangleright \outb(a) \rightarrow t \qquad \text{ if } \Gamma \triangleright a \uparrow A[\psi \to t] \text{ and } \Gamma \vdash \psi \\
      \Gamma \triangleright \outb(in(a)) \rightarrow a \\
      \Gamma \triangleright \outb(a) \rightarrow a' \qquad \text{ if } \Gamma \triangleright a \rightarrow a' \\
      \Gamma \triangleright \outp(in(a)) \rightarrow a \\
      \Gamma \triangleright \outp(a) \rightarrow a' \qquad \text{ if  } \Gamma \triangleright a \rightarrow a'\\
    \end{array}
  \]
\end{definition}


\begin{definition}
  Term equivalence is defined as follows:

  \[
    \begin{array}{lr}
      \Gamma \triangleright M \Leftrightarrow N : A[\psi \to t] \qquad \text{ if } \Gamma \triangleright \outb(M) \Leftrightarrow \outb(N) : A\\
      \Gamma \triangleright M \Leftrightarrow N : [\alpha] \to A \qquad \text{ if } \Gamma, \alpha \triangleright \outp(M) \Leftrightarrow \outp(N) : A\\
      \Gamma \triangleright M \Leftrightarrow N : (x : \mathbb{I}) \to A \qquad \text{ if } \Gamma, x : \mathbb{I} \triangleright M x \Leftrightarrow N x : A(x)\\
      \Gamma \triangleright M \Leftrightarrow N : (x : \cof) \to A \qquad \text{ if } \Gamma, x : \cof \triangleright M x \Leftrightarrow N x : A(x)\\
    \end{array}
  \]
\end{definition}

\begin{definition}
  Neutral equivalence is defined as follows:

  \[
    \begin{array}{lr}
      \Gamma \triangleright [\alpha \to p_{1}, \beta \to p_{2}] \leftrightarrow q \uparrow A \qquad \text{ if } \Gamma, \alpha \triangleright p_{1} \Leftrightarrow q : A \text{ and } \Gamma, \beta \triangleright p_{2} \Leftrightarrow q : A \\

      \Gamma \triangleright p \leftrightarrow [\alpha \to q_{1}, \beta \to q_{2}] \uparrow A \qquad \text{ if } \Gamma, \alpha \triangleright p \Leftrightarrow q_{1} : A \text{ and } \Gamma, \beta \triangleright p \Leftrightarrow q_{2} : A \\

      \Gamma \triangleright \outb(p) \Leftrightarrow \outb(q) \uparrow A \qquad \text{ if } \Gamma \triangleright p \leftrightarrow q \uparrow A[\psi \to t]\\

      \Gamma \triangleright \outp(p) \Leftrightarrow \outb(q) \uparrow A \qquad \text{ if } \Gamma \triangleright p \leftrightarrow q \uparrow [\alpha] \to A\\

    \end{array}
  \]
\end{definition}

\end{document}
